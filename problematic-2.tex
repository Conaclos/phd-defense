
\begin{frame}[plain,standout]
    Problématique 2~:\\peut-on concevoir un protocole pour la co-édition de texte qui résiste mieux aux aléas du réseau et aux longues périodes de déconnexion~?\\
    \vspace{2em}
    \begin{tikzpicture}[x=1.24cm]
        \node at (-1,0) {\includegraphics[scale=1]{collab/writer.pdf}};
        \node[xscale=-1] at (1,0) {\includegraphics[scale=1]{collab/writer.pdf}};
    \end{tikzpicture}
\end{frame}


\begin{frame}{Édition collaborative de texte}
    \begin{minipage}[c][.55\textheight][t]{\textwidth}
        \begin{tikzpicture}[x=\textwidth,y=1.2cm]
            \node(A)[
                anchor=north west,
                "Alice" {username,above},
                text width=7*\lettersize,
                label={[align=left,text width=3cm]below:
                    {
                        \only<2->{\\$\trm{ins}(3,i)$}
                        \only<3->{\\$\trm{del}(5)$}
                        \only<4->{\\$\trm{ins}(4,e)$}
                        \only<4->{\\$\trm{ins}(5,s)$}
                        \only<5->{\\$\trm{ins}(4,!)$}
                    }
                }
            ] {\begin{tikzpicture}[
                graphs/every graph/.style={
                    nodes={letter},
                    edges={draw=none},
                    grow right=\lettersize
                }
            ]
                \only<1>{
                    \graph{
                        l -- a -- d -- y
                    };
                }
                \only<2>{
                    \graph{
                        l -- a -- d -- i[fillhighlight] -- y
                    };
                }
                \only<3>{
                    \graph{
                        l -- a -- d -- i -- y[as={\st{y}}]
                    };
                }
                \only<4>{
                    \graph{
                        l -- a -- d -- i -- e[fillhighlight] -- s[fillhighlight]
                    };
                }
                \only<5>{
                    \graph{
                        l -- a -- d -- i -- ![fillhighlight] -- e -- s
                    };
                }
            \end{tikzpicture}};

            \node(B) at (1,0)[
                anchor=north east,
                "Bea" {username,above},
                text width=7*\lettersize,
                label={[align=left,text width=3cm]below:
                    {
                        \only<4->{\\$\trm{ins}(4,!)$}
                        \only<5->{
                            \\$\trm{ins}(3,i)$
                            \\$\trm{del}(5)$
                            \\$\trm{ins}(4,e)$
                            \\$\trm{ins}(5,s)$
                        }
                    }
                }
            ] {\begin{tikzpicture}[
                graphs/every graph/.style={
                    nodes={letter},
                    edges={draw=none},
                    grow right=\lettersize
                }
            ]
                \only<1-3>{
                    \graph{
                        l -- a -- d -- y
                    };
                }
                \only<4>{
                    \graph{
                        l -- a -- d -- y -- ![fillhighlight]
                    };
                }
                \only<5>{
                    \graph{
                        l -- a -- d -- i[fillhighlight] -- e[fillhighlight] -- s[fillhighlight] -- !
                    };
                }
            \end{tikzpicture}};
        \end{tikzpicture}
    \end{minipage}
    \begin{minipage}{\textwidth}
        \begin{compactitemize}
            \item document = \textbf{séquence} de caractères
            \begin{compactitemize}
                \item $\trm{ins}(c,n)$~: insertion d'un caractère $c$ après le $n$-ième caractère
                \item $\trm{del}(n)$~: suppression du $n$-ième caractère
            \end{compactitemize}
        \end{compactitemize}
    \end{minipage}
\end{frame}


\begin{frame}{Types de données répliquées}
    \begin{minipage}[c][.55\textheight][t]{\textwidth}
        \begin{tikzpicture}[x=\textwidth,y=1.2cm]
            \node(A)[
                anchor=north west,
                "Alice" {username,above},
                text width=7*\lettersize,
                label={[align=left,text width=3cm]below:
                    {
                        \\$\trm{ins}(3,i)$
                        \\$\trm{del}(5)$
                        \\$\trm{ins}(4,e)$
                        \\$\trm{ins}(5,s)$
                        \only<2>{\\message de Bea}
                    }
                }
            ] {\begin{tikzpicture}[
                graphs/every graph/.style={
                    nodes={letter},
                    edges={draw=none},
                    grow right=\lettersize
                }
            ]
                \only<1>{
                    \graph{
                        l -- a -- d -- i -- e -- s
                    };
                }
                \only<2>{
                    \graph{
                        l -- a -- d -- i  -- e -- s -- ![fillhighlight]
                    };
                }
            \end{tikzpicture}};

            \node(B) at (1,0)[
                anchor=north east,
                "Bea" {username,above},
                text width=7*\lettersize,
                label={[align=left,text width=3cm]below:
                    {
                        \\$\trm{ins}(4,!)$
                        \only<2>{\\message·s de Alice}
                    }
                }
            ] {\begin{tikzpicture}[
                graphs/every graph/.style={
                    nodes={letter},
                    edges={draw=none},
                    grow right=\lettersize
                }
            ]
                \only<1>{
                    \graph{
                        l -- a -- d -- y -- !
                    };
                }
                \only<2>{
                    \graph{
                        l -- a -- d -- i[fillhighlight] -- e[fillhighlight] -- s[fillhighlight] -- !
                    };
                }
            \end{tikzpicture}};
        \end{tikzpicture}
    \end{minipage}
    \begin{minipage}{\textwidth}
        \begin{compactitemize}
            \item Conflict-free Replicated Data Types (CRDTs)
            \item l'exécution d'une opération génère un \textbf{message de synchronisation}
            \item les messages générés par des \textbf{opérations concurrentes} peuvent être \textbf{intégrés dans un ordre quelconque}
        \end{compactitemize}
    \end{minipage}
\end{frame}


\begin{frame}{Synchronisation par opérations (1)}
    \begin{minipage}[c][.55\textheight][t]{\textwidth}
        \centering
        \begin{tikzpicture}
            \newcommand*\sep{1.5}
            \newcommand*\ang{80}
            % devices
            \node(A) at (\ang:\sep)[
                "Alice",
                label=right:{%
                    \tikz[scale=.05]{
                        \pic {doc};
                        \pic[fill=sky] at (1,13) {square};
                        % \pic[fill=olive] at (7,8) {circle};
                        % \pic[fill=pinky] at (7,1) {triangle};
                    }
                },
                label={[shift={(1,0)},visible on=<2>]right:{
                    \color{invalid} \xmark
                }},
            ]{\includegraphics[scale=.35]{collab/device.pdf}};
            \node(B) at (\ang-120:\sep)[
                "Bea" below,
                label=right:{%
                    \tikz[scale=.05]{
                        \pic {doc};
                        %\pic[fill=sky] at (1,13) {square};
                        \pic[fill=olive] at (7,8) {circle};
                        \pic<2 | handout:0>[fill=pinky] at (7,1) {triangle};
                    }
                },
                label={[shift={(1,0)},visible on=<2>]right:{
                    \color{valid} \checkmark
                }},
            ]{\includegraphics[scale=.35]{collab/device.pdf}};
            \node(C) at (\ang+120:\sep)[
                "Carol" below,
                label=left:{%
                    \tikz[scale=.05]{
                        \pic {doc};
                        %\pic[fill=sky] at (1,13) {square};
                        \pic[fill=olive] at (7,8) {circle};
                        \pic[fill=pinky] at (7,1) {triangle};
                    }
                },
            ]{\includegraphics[scale=.35]{collab/device.pdf}};
            % Links
            \draw[link] (A) -- (B);
            \draw<1>[link,->] (C) -> node[midway]{\tikz\pic[fill=pinky]{triangle};} (A);
            \draw<1>[link,->] (C) -> node[midway]{\tikz\pic[fill=pinky]{triangle};} (B);
            \draw<2>[link] (C) -> (A);
            \draw<2>[link] (C) -> (B);
        \end{tikzpicture}
    \end{minipage}
    \begin{minipage}{\textwidth}
        \begin{compactenumerate}
            \item un message est intégré \textbf{exactement une fois}
            \item ordre spécifique d'intégration de certains messages
            \begin{compactitemize}
                \item l'\textbf{insertion} d'une valeur est intégrée \textbf{avant sa suppression}
            \end{compactitemize}
            \item \textbf{intégration causale} des messages (implique 2.)
        \end{compactenumerate}
    \end{minipage}
\end{frame}


\begin{frame}{Synchronisation par opérations (2)}
    \begin{minipage}[c][.55\textheight][t]{\textwidth}
        \centering
        \begin{tikzpicture}
            \newcommand*\sep{1.5}
            \newcommand*\ang{80}
            % devices
            \node(A) at (\ang:\sep)[
                "Alice",
                label=right:{%
                    \tikz[scale=.05]{
                        \pic {doc};
                        \pic[fill=sky] at (1,13) {square};
                        \pic[fill=olive] at (7,8) {circle};
                        \pic[fill=pinky] at (7,1) {triangle};
                    }
                },
            ]{\includegraphics[scale=.35]{collab/device.pdf}};
            \node(B) at (\ang-120:\sep)[
                "Bea" below,
                label=right:{%
                    \tikz[scale=.05]{
                        \pic {doc};
                        \pic[fill=sky] at (1,13) {square};
                        \pic[fill=olive] at (7,8) {circle};
                        \pic[fill=pinky] at (7,1) {triangle};
                    }
                },
            ]{\includegraphics[scale=.35]{collab/device.pdf}};
            \node(C) at (\ang+120:\sep)[
                "Carol" below,
                label=left:{%
                    \tikz[scale=.05]{
                        \pic {doc};
                        \pic[fill=sky] at (1,13) {square};
                        \pic[fill=olive] at (7,8) {circle};
                        \pic[fill=pinky] at (7,1) {triangle};
                    }
                },
            ]{\includegraphics[scale=.35]{collab/device.pdf}};
            \coordinate(Dp) at (\ang+60:3*\sep);
            \node(D) at (Dp |- A) [
                "Diane",
                label=left:{%
                    \tikz[scale=.05]{
                        \pic {doc};
                        \pic<2->[fill=sky] at (1,13) {square};
                        \pic<2->[fill=olive] at (7,8) {circle};
                        \pic<2->[fill=pinky] at (7,1) {triangle};
                        \pic[fill=gold] at (1,1) {trapeze};
                    }
                },
            ]{\includegraphics[scale=.35]{collab/device.pdf}};
            % Links
            \draw[link] (A) -- (B);
            \draw[link] (C) -- (A);
            \draw[link] (C) -> (B);
            \draw<1>[link,->] (C) ->
                node[pos=.75]{\tikz\pic[fill=sky]{square};}
                node[pos=.5]{\tikz\pic[fill=olive]{circle};}
                node[pos=.3]{\tikz\pic[fill=pinky]{triangle};}
                (D);
            \draw<2>[link] (C) -> (D);
        \end{tikzpicture}
    \end{minipage}
    \begin{minipage}{\textwidth}
        \begin{compactitemize}
            \item[\color{valid} \textbf{\texttt{+}}] messages de petite taille
            \item[\color{invalid} \textbf{\texttt{-}}] exige une dissémination fiable des message
            \item[\color{invalid} \textbf{\texttt{-}}] la perte d'un message peut \textbf{propager des ralentissements}
            \item[\color{invalid} \textbf{\texttt{-}}] coût en communication et en calcul élevé
            \begin{compactitemize}
                \item après une \textbf{longue période de déconnexion}
            \end{compactitemize}
        \end{compactitemize}
    \end{minipage}
\end{frame}


\begin{frame}{Synchronisation par états (1)}
    \begin{minipage}[c][.55\textheight][t]{\textwidth}
        \centering
        \begin{tikzpicture}
            \newcommand*\sep{1.5}
            \newcommand*\ang{80}
            % devices
            \node(A) at (\ang:\sep)[
                "Alice",
                label=right:{%
                    \tikz[scale=.05]{
                        \pic {doc};
                        \pic[fill=sky] at (1,13) {square};
                        \pic<2 | handout:0>[fill=olive] at (7,8) {circle};
                        \pic<2 | handout:0>[fill=pinky] at (7,1) {triangle};
                    }
                },
            ]{\includegraphics[scale=.35]{collab/device.pdf}};
            \node(B) at (\ang-120:\sep)[
                "Bea" below,
                label=right:{%
                    \tikz[scale=.05]{
                        \pic {doc};
                        %\pic[fill=sky] at (1,13) {square};
                        \pic[fill=olive] at (7,8) {circle};
                        \pic<2 | handout:0>[fill=pinky] at (7,1) {triangle};
                    }
                },
            ]{\includegraphics[scale=.35]{collab/device.pdf}};
            \node(C) at (\ang+120:\sep)[
                "Carol" below,
                label=left:{%
                    \tikz[scale=.05]{
                        \pic {doc};
                        %\pic[fill=sky] at (1,13) {square};
                        \pic[fill=olive] at (7,8) {circle};
                        \pic[fill=pinky] at (7,1) {triangle};
                    }
                },
            ]{\includegraphics[scale=.35]{collab/device.pdf}};
            % Links
            \draw[link] (A) -- (B);
            \draw<1>[link,->] (C) -> node[midway]{\tikz[scale=.05]{
                \pic {doc};
                %\pic[fill=sky] at (1,13) {square};
                \pic[fill=olive] at (7,8) {circle};
                \pic[fill=pinky] at (7,1) {triangle};
            }} (A);
            \draw<1>[link,->] (C) -> node[midway]{\tikz[scale=.05]{
                \pic {doc};
                %\pic[fill=sky] at (1,13) {square};
                \pic[fill=olive] at (7,8) {circle};
                \pic[fill=pinky] at (7,1) {triangle};
            }} (B);
            \draw<2>[link] (C) -> (A);
            \draw<2>[link] (C) -> (B);
        \end{tikzpicture}
    \end{minipage}
    \begin{minipage}{\textwidth}
        \begin{compactitemize}
            \item les pairs \textbf{transmettent} directement l'\textbf{état mis à jour}
            \item ils \textbf{fusionnent} leur état actuel avec les états reçus
        \end{compactitemize}
    \end{minipage}
\end{frame}


\begin{frame}{Synchronisation par états (2)}
    \begin{minipage}[c][.7\textheight][t]{\textwidth}
        \centering
        \begin{tikzpicture}[
            x={(.75cm,.5cm)}, y={(-.75cm,.5cm)}, z={(0cm,.55cm)},
            scale=3,
            every node/.style={inner sep=.222em},
        ]
            \newcommand\doc{\tikz[scale=.05,x=1cm,y=1cm]{\pic{doc};}}
            \newcommand\docA{\tikz[scale=.05,x=1cm,y=1cm]{\pic{doc};\pic[fill=sky] at (1,13) {square};}}
            \newcommand\docB{\tikz[scale=.05,x=1cm,y=1cm]{\pic {doc};\pic[fill=olive] at (7,8) {circle};}}
            \newcommand\docC{\tikz[scale=.05,x=1cm,y=1cm]{\pic{doc};\pic[fill=pinky] at (7,1) {triangle};}}
            \newcommand\docAB{\tikz[scale=.05,x=1cm,y=1cm]{\pic{doc};\pic[fill=sky] at (1,13) {square};\pic[fill=olive] at (7,8) {circle};}}
            \newcommand\docAC{\tikz[scale=.05,x=1cm,y=1cm]{\pic{doc};\pic[fill=sky] at (1,13) {square};\pic[fill=pinky] at (7,1) {triangle};}}
            \newcommand\docBC{\tikz[scale=.05,x=1cm,y=1cm]{\pic{doc};\pic[fill=olive] at (7,8) {circle};\pic[fill=pinky] at (7,1) {triangle};}}
            \newcommand\docABC{\tikz[scale=.05,x=1cm,y=1cm]{\pic{doc};\pic[fill=sky] at (1,13) {square};\pic[fill=olive] at (7,8) {circle};\pic[fill=pinky] at (7,1) {triangle};}}

            \node(000) at (0,0,0){\doc};
            \node(100) at (1,0,0)[fillhighlight on=<1>]{\docC};
            \node(010) at (0,1,0){\docA};
            \node(001) at (0,0,1)[fillhighlight on=<{1,3}>]{\docB};
            \node(011) at (0,1,1){\docAB};
            \node(101) at (1,0,1)[fillhighlight on=<2-4>]{\docBC};
            \node(110) at (1,1,0){\docAC};
            \node(111) at (1,1,1){\docABC};
            \graph[edges={opacity=.7}]{
                (100) ->[opacity=.4] (110),
                (010) ->[opacity=.4] (110),
                (110) ->[opacity=.4] (111),
                (000) -> (100),
                (000) -> (010),
                (000) -> (001),
                (100) ->[on={<1>}{very thick,opacity=1}] (101),
                (010) -> (011),
                (001) ->[on={<{1,3}>}{very thick,opacity=1}] (101),
                (001) -> (011),
                (101) -> (111),
                (011) -> (111),
            };
        \end{tikzpicture}
    \end{minipage}
    \begin{minipage}{\textwidth}
        \begin{compactitemize}
            \item l'ensemble des états forme un \textbf{sup demi-treillis}
            \begin{compactitemize}
                \item fusion de 2 états donne le plus petit état qui leur est $\geq$
                \item fusion~: commutative, associative, idempotente
            \end{compactitemize}
        \end{compactitemize}
    \end{minipage}
\end{frame}


\begin{frame}{Synchronisation par états (3)}
    \begin{minipage}[c][.55\textheight][t]{\textwidth}
        \centering
        \begin{tikzpicture}
            \newcommand*\sep{1.5}
            \newcommand*\ang{80}
            % devices
            \node(A) at (\ang:\sep)[
                "Alice",
                label=right:{%
                    \tikz[scale=.05]{
                        \pic {doc};
                        \pic[fill=sky] at (1,13) {square};
                        \pic<5 | handout:0>[fill=olive] at (7,8) {circle};
                        \pic<5 | handout:0>[fill=pinky] at (7,1) {triangle};
                    }
                },
            ]{\includegraphics[scale=.35]{collab/device.pdf}};
            \node(B) at (\ang-120:\sep)[
                "Bea" below,
                label=right:{%
                    \tikz[scale=.05]{
                        \pic {doc};
                        %\pic[fill=sky] at (1,13) {square};
                        \pic[fill=olive] at (7,8) {circle};
                        \pic<5 | handout:0>[fill=pinky] at (7,1) {triangle};
                    }
                },
            ]{\includegraphics[scale=.35]{collab/device.pdf}};
            \node(C) at (\ang+120:\sep)[
                "Carol" below,
                label=left:{%
                    \tikz[scale=.05]{
                        \pic {doc};
                        %\pic[fill=sky] at (1,13) {square};
                        \pic<2->[fill=olive] at (7,8) {circle};
                        \pic<3->[fill=pinky] at (7,1) {triangle};
                    }
                },
            ]{\includegraphics[scale=.35]{collab/device.pdf}};
            % Links
            \draw<1-5>[link] (A) -- node[midway]{\color{invalid} \xmark} (B);
            \draw<4>[link,->] (C) -> node[midway]{\tikz[scale=.05]{
                \pic {doc};
                %\pic[fill=sky] at (1,13) {square};
                \pic[fill=olive] at (7,8) {circle};
                \pic[fill=pinky] at (7,1) {triangle};
            }} (A);
            \draw<1>[link,->] (B) -> node[midway]{\tikz[scale=.05]{
                \pic {doc};
                \pic[fill=olive] at (7,8) {circle};
            }} (C);
            \draw<4>[link,->] (C) -> node[midway]{\tikz[scale=.05]{
                \pic {doc};
                %\pic[fill=sky] at (1,13) {square};
                \pic[fill=olive] at (7,8) {circle};
                \pic[fill=pinky] at (7,1) {triangle};
            }} (B);
            \draw<1-3,5>[link] (C) -> (A);
            \draw<2-3,5>[link] (C) -> (B);
        \end{tikzpicture}
    \end{minipage}
    \begin{minipage}{\textwidth}
        \begin{compactitemize}
            \item[\color{valid} \textbf{\texttt{+}}] \textbf{résiste aux aléas du réseau}
            \begin{compactitemize}
                \item tolère perte, duplication, et réordonnement des messages
            \end{compactitemize}
            \item[\color{invalid} \textbf{\texttt{-}}] adapté aux \textbf{CRDTs avec une faible occupation mémoire}
        \end{compactitemize}
    \end{minipage}
\end{frame}


\begin{frame}{Synchronisation par différences (d'états)}
    \begin{minipage}[c][.55\textheight][t]{\textwidth}
        \centering
        \begin{tikzpicture}
            \newcommand*\sep{1.5}
            \newcommand*\ang{80}
            % devices
            \node(A) at (\ang:\sep)[
                "Alice",
                label=right:{%
                    \tikz[scale=.05]{
                        \pic {doc};
                        \pic[fill=sky] at (1,13) {square};
                        \pic<2 | handout:0>[fill=olive] at (7,8) {circle};
                        \pic<2 | handout:0>[fill=pinky] at (7,1) {triangle};
                    }
                },
            ]{\includegraphics[scale=.35]{collab/device.pdf}};
            \node(B) at (\ang-120:\sep)[
                "Bea" below,
                label=right:{%
                    \tikz[scale=.05]{
                        \pic {doc};
                        %\pic[fill=sky] at (1,13) {square};
                        \pic[fill=olive] at (7,8) {circle};
                        \pic<2 | handout:0>[fill=pinky] at (7,1) {triangle};
                    }
                },
            ]{\includegraphics[scale=.35]{collab/device.pdf}};
            \node(C) at (\ang+120:\sep)[
                "Carol" below,
                label=left:{%
                    \tikz[scale=.05]{
                        \pic {doc};
                        %\pic[fill=sky] at (1,13) {square};
                        \pic[fill=olive] at (7,8) {circle};
                        \pic[fill=pinky] at (7,1) {triangle};
                    }
                },
            ]{\includegraphics[scale=.35]{collab/device.pdf}};
            % Links
            \draw[link] (A) -- (B);
            \draw<1>[link,->] (C) -> node[midway]{\tikz[scale=.05]{
                \pic {doc};
                \pic[fill=olive] at (7,8) {circle};
                \pic[fill=pinky] at (7,1) {triangle};
            }} (A);
            \draw<1>[link,->] (C) -> node[midway]{\tikz[scale=.05]{
                \pic {doc};
                \pic[fill=pinky] at (7,1) {triangle};
            }} (B);
            \draw<2>[link] (C) -> (A);
            \draw<2>[link] (C) -> (B);
        \end{tikzpicture}
    \end{minipage}
    \begin{minipage}{\textwidth}
        \begin{compactitemize}
            \item un état est \textbf{décomposable en états irréductibles}
            \item différence entre l'état initial et l'état mis à jour
            \item[\color{valid} \textbf{\texttt{+}}] flexibilité pour la synchronisation
            \begin{compactitemize}
                \item synchronisation par états
                \item \textbf{synchronisation par états partiels} (différences d'états)
            \end{compactitemize}
        \end{compactitemize}
    \end{minipage}
\end{frame}


\begin{frame}{Logoot\footcite{weiss_2009_logoot}}
    \begin{minipage}[c][.33\textheight][t]{\textwidth}
        \begin{tikzpicture}[x=\textwidth]
            \node(A)[
                anchor=north west,
                "Alice" {username,above},
                text width=7*\lettersize,
            ] {\begin{tikzpicture}[
                graphs/every graph/.style={
                    nodes={letter},
                    edges={draw=none},
                    grow right=\lettersize
                },
            ]
                \only<1>{
                    \graph{
                        l -- a -- d -- y
                    };
                }
                \only<2>{
                    \graph{
                        l -- a -- d -- i[fillhighlight] -- y
                    };
                }
                \only<3>{
                    \graph{
                        l -- a -- d -- i -- e[fillhighlight] -- s[fillhighlight]
                    };
                }
                \only<4->{
                    \graph{
                        l -- a -- d -- i  -- e -- s -- ![fillhighlight]
                    };
                }
                \only<1-4>{
                    \node also["$f$" below] (l);
                    \node also["$h$" below] (a);
                    \node also["$l$" below] (d);
                    \node<-2> also["$m$" below] (y);
                    \node<2-> also["$lq$" below] (i);
                    \node<3-> also["$p$" below] (e);
                    \node<3-> also["$r$" below] (s);
                    \node<4-> also["$t$" below] (!);
                }
                \only<5>{
                    \node also["$f^{A1}$" below] (l);
                    \node also["$h^{A2}$" below] (a);
                    \node also["$l^{A3}$" below] (d);
                    %\node also["$m^{A4}$" below] (y);
                    \node also["$lq^{A5}$" below] (i);
                    \node also["$p^{A6}$" below] (e);
                    \node also["$r^{A7}$" below] (s);
                    \node also["$t^{B1}$" below] (!);
                }
            \end{tikzpicture}};

            \node(B) at (1,0)[
                anchor=north east,
                "Bea" {username,above},
                text width=7*\lettersize,
            ] {\begin{tikzpicture}[
                graphs/every graph/.style={
                    nodes={letter},
                    edges={draw=none},
                    grow right=\lettersize
                }
            ]
                \only<1-2>{
                    \graph{
                        l -- a -- d -- y
                    };
                }
                \only<3>{
                    \graph{
                        l -- a -- d -- y -- ![fillhighlight]
                    };
                }
                \only<4->{
                    \graph{
                        l -- a -- d -- i[fillhighlight] -- e[fillhighlight] -- s[fillhighlight] -- !
                    };
                }
                \only<1-4>{
                    \node also["$f$" below] (l);
                    \node also["$h$" below] (a);
                    \node also["$l$" below] (d);
                    \node<-3> also["$m$" below] (y);
                    \node<4-> also["$lq$" below] (i);
                    \node<4-> also["$p$" below] (e);
                    \node<4-> also["$r$" below] (s);
                    \node<3-> also["$t$" below] (!);
                }
                \only<5>{
                    \node also["$f^{A1}$" below] (l);
                    \node also["$h^{A2}$" below] (a);
                    \node also["$l^{A3}$" below] (d);
                    %\node also["$m^{A4}$" below] (y);
                    \node also["$lq^{A5}$" below] (i);
                    \node also["$p^{A6}$" below] (e);
                    \node also["$r^{A7}$" below] (s);
                    \node also["$t^{B1}$" below] (!);
                }
            \end{tikzpicture}};
        \end{tikzpicture}
    \end{minipage}
    \begin{minipage}{\textwidth}
        \begin{compactitemize}
            \item chaque caractère à une \textbf{position unique} et immuable
            \item \textbf{ordre total et dense entre les positions}
            \begin{compactitemize}
                \item une position peut toujours être générée entre deux autres
            \end{compactitemize}
            \item chaque caractère à un \textbf{identifiant unique} (\emph{dot})
            \begin{compactitemize}
                \item identifiant du pair qui l'a inséré
                \item entier incrémenté avant chaque insertion du pair
            \end{compactitemize}
        \end{compactitemize}
    \end{minipage}
\end{frame}


\begin{frame}{Contribution~: Logoot synchronisé par différences}
    \begin{minipage}[c][.55\textheight][t]{\textwidth}
        \begin{tikzpicture}[x=\textwidth,y=-2cm]
            \node(ladiy!) at (.5,0) {\begin{tikzpicture}[every node/.style={bigletter},every label quotes/.style={below,align=center}]
                \node(l){l};
                \node(a)[right of=l]{a};
                \node(d)[right of=a]{d};
                \node(i)[fillhighlight, right of=d]{i};
                \node(y)[right of=i]{y};
                \node(!)[fillhighlight, right of=y]{!};
                \node also["$f^{A1}$"] (l);
                \node also["$h^{A2}$"] (a);
                \node also["$l^{A3}$"] (d);
                \node also["$lq^{A5}$"] (i);
                \node also["$m^{A4}$"] (y);
                \node also["$t^{B1}$"] (!);
            \end{tikzpicture}};

            \node(ladiy)[anchor=west] at (0,1) {\begin{tikzpicture}[every node/.style={bigletter},every label quotes/.append style={below}]
                \node(l){l};
                \node(a)[right of=l]{a};
                \node(d)[right of=a]{d};
                \node(i)[fillhighlight, right of=d]{i};
                \node(y)[right of=i]{y};
                \node also["$f^{A1}$"] (l);
                \node also["$h^{A2}$"] (a);
                \node also["$l^{A3}$"] (d);
                \node also["$lq^{A5}$"] (i);
                \node also["$m^{A4}$"] (y);
            \end{tikzpicture}};

            \node(lady!)[anchor=east] at (1,1) {\begin{tikzpicture}[every node/.style={bigletter},every label quotes/.style={below}]
                \node(l){l};
                \node(a)[right of=l]{a};
                \node(d)[right of=a]{d};
                \node(y)[right of=d]{y};
                \node(!)[fillhighlight, right of=y]{!};
                \node also["$f^{A1}$"] (l);
                \node also["$h^{A2}$"] (a);
                \node also["$l^{A3}$"] (d);
                \node also["$m^{A4}$"] (y);
                \node also["$t^{B1}$"] (!);
            \end{tikzpicture}};

            \graph {
                (ladiy.north) -> (ladiy!.west),
                (lady!.north) -> (ladiy!.east)
            };
        \end{tikzpicture}
    \end{minipage}
    \begin{minipage}{\textwidth}
        \begin{compactitemize}
            \item \textbf{\emph{Logoot} sans suppression} synchronisé par différences d'états
            \begin{compactitemize}
                \item union des ensembles de pairs valeur-position
            \end{compactitemize}
            \item valeur supprimée $\cong$ valeur qui n'a pas encore été insérée
        \end{compactitemize}
    \end{minipage}
\end{frame}


\begin{frame}{Contribution~: séquence synchronisée par différences}
    \begin{minipage}[c][.55\textheight][t]{0.62\textwidth}
        \centering
        \begin{tikzpicture}[
            x={(.866cm,.5cm)}, y={(-.866cm,.5cm)}, z={(0cm,.63cm)},
            scale=2.5,
            every node/.style={compact,minimum height=5mm,minimum width=5mm,font=\ttfamily},
        ]
            \node(000) at (0,0,0) {y};
            \node(100) at (1,0,0) {!};
            \node(010)[fillhighlight on=<1>] at (0,1,0) {lady};
            \node(110)[opacity=.7,fillhighlight on=<2>] at (1,1,0) {lady!};
            \node(001) at (0,0,1) {\st{y}};
            \node(011) at (0,1,1) {lad\st{y}};
            \node(101) at (1,0,1) {\st{y}!};
            \node(111)[fillhighlight on=<3>] at (1,1,1) {lad\st{y}!};
            \graph[nodes={opacity=.7}]{
                (100) ->[opacity=.4] (110),
                (010) ->[opacity=.4,on={<2-3>}{very thick,opacity=1,"1" sloped}] (110),
                (110) ->[opacity=.4,on={<3>}{very thick,opacity=1,"2" sloped}] (111),
                (000) -> (100),
                (000) -> (010),
                (000) -> (001),
                (100) -> (101),
                (010) -> (011),
                (001) -> (101),
                (001) -> (011),
                (101) -> (111),
                (011) -> (111),
            };
        \end{tikzpicture}
    \end{minipage}
    \begin{minipage}[c][.55\textheight][t]{0.26\textwidth}
        \vspace{3em}
        \begin{compactenumerate}
            \item intègre \texttt{\textquotesingle!\textquotesingle}
            \item intègre \texttt{\textquotesingle\st{y}\textquotesingle}
        \end{compactenumerate}
    \end{minipage}
    \begin{minipage}{\textwidth}
        \begin{compactitemize}
            \item deux états irréductibles pour chaque couple caractère-position
            \begin{compactitemize}
                \item un qui représente la présence du couple
                \item un qui représente la suppression du couple
                \item le premier est plus petit que le second
            \end{compactitemize}
        \end{compactitemize}
    \end{minipage}
\end{frame}


\begin{frame}{Contribution~: Logoot synchronisé par différences}
    \begin{minipage}[c][.6\textheight][t]{\textwidth}
        \begin{tikzpicture}[x=\textwidth,y=-2.7cm]
            \node(ladi!)[
                "$\set*{A1, A2, A3, \textbf{A4}, A5, B1}$",
            ] at (.5,0) {\begin{tikzpicture}[every node/.style={bigletter},every label quotes/.style={below,align=center}]
                \node(l){l};
                \node(a)[right of=l]{a};
                \node(d)[right of=a]{d};
                \node(i)[fillhighlight, right of=d]{i};
                \node(!)[fillhighlight, right of=i]{!};
                \node also["$f^{A1}$"] (l);
                \node also["$h^{A2}$"] (a);
                \node also["$l^{A3}$"] (d);
                \node also["$lq^{A5}$"] (i);
                \node also["$t^{B1}$"] (!);
            \end{tikzpicture}};

            \node(ladi)[
                anchor=west,
                "$\set*{A1, A2, A3, \textbf{A4}, A5}$" {name={ladi label}},
            ] at (0,1) {\begin{tikzpicture}[every node/.style={bigletter},every label quotes/.append style={below}]
                \node(l){l};
                \node(a)[right of=l]{a};
                \node(d)[right of=a]{d};
                \node(i)[fillhighlight, right of=d]{i};
                \node also["$f^{A1}$"] (l);
                \node also["$h^{A2}$"] (a);
                \node also["$l^{A3}$"] (d);
                \node also["$lq^{A5}$"] (i);
            \end{tikzpicture}};

            \node(lady!)[
                anchor=east,
                "$\set*{A1, A2, A3, \textbf{A4}, B1}$" {name={lady! label}},
            ] at (1,1) {\begin{tikzpicture}[every node/.style={bigletter},every label quotes/.style={below}]
                \node(l){l};
                \node(a)[right of=l]{a};
                \node(d)[right of=a]{d};
                \node(y)[right of=d]{y};
                \node(!)[fillhighlight, right of=y]{!};
                \node also["$f^{A1}$"] (l);
                \node also["$h^{A2}$"] (a);
                \node also["$l^{A3}$"] (d);
                \node also["$m^{\textbf{A4}}$"] (y);
                \node also["$t^{B1}$"] (!);
            \end{tikzpicture}};

            \graph {
                (ladi label.north) -> (ladi!.west),
                (lady! label.north) -> (ladi!.east)
            };
        \end{tikzpicture}
    \end{minipage}
    \begin{minipage}[t]{\textwidth}
        \begin{compactitemize}
            \item état ~:
            \begin{compactitemize}
                \item l'ensemble des valeurs et leurs positions
                \item l'ensemble des identifiants des valeurs insérées
            \end{compactitemize}
            \item le deuxième ensemble évite la réinsertion d'une valeur déjà insérée
        \end{compactitemize}
    \end{minipage}
\end{frame}


\begin{frame}{Résumé des contributions}
    \begin{compactitemize}
        \item contributions
        \begin{compactitemize}
            \item \textbf{formalisation des séquences} à positions densément ordonnées
            \item synchronisation par différences d'état de ces séquences (e.g.\ \emph{Logoot})
            \item nouvelle structure de positions~: \textbf{Dotted LogootSplit}
            \begin{compactitemize}
                \item \textbf{optimisée} pour la synchronisation par différences d'états
                \item \textbf{positions agrégeables} pour former des blocs
                \item occupent moins d'espace mémoire que les positions LogootSplit
            \end{compactitemize}
        \end{compactitemize}
        \item notre approche
        \begin{compactitemize}
            \item \textbf{résiste} mieux aux \textbf{aléas du réseau}
            \item peut être appliquée à plusieurs séquences répliquées
            \item implémentée\singlefootnote{\url{https://github.com/coast-team/dotted-logootsplit}} et testée en conditions réelles
        \end{compactitemize}
    \end{compactitemize}
\end{frame}

